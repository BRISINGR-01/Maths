\documentclass[main.tex]{subfiles}
\begin{document}

\subsection{Quantifiable control performance metrics}
In order to compare the two approaches - LQR and PID, and their overall optimization quality, we will use specific measurable criteria that determine how well the system behaves after a disturbance in terms of reaching a steady state.

\begin{enumerate}
  \item \textbf{Stability} - The ability of the system to return to its equilibrium state after a disturbance. \cite{ogata}
  \item \textbf{Settling time} - The time needed for the system to settle within a certain percentage of its final value.\cite{ogata}
  \item \textbf{Overshoot} - The maximum amount by which the system's response exceeds its final value.\cite{ogata}
  \item \textbf{Control effort} - The actuator demand required to stabilize the system, quantified in terms of coil current. \cite{ctrlefrt}
  \item \textbf{RMS current} - captures the average intensity of controller operation during simulation. \cite{rmspeak}
  \item \textbf{Peak current} - captures the highest possible current load during operation \cite{rmspeak}
\end{enumerate}


\end{document}
