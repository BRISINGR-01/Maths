\documentclass[main.tex]{subfiles}
\begin{document}
\section{Conclusions}

In this project, we studied and applied fundamental concepts from classical and modern control theory using a magnetically actuated inverted pendulum as a case study. The work provided practical insight into magnetic actuation, a technology that is increasingly relevant in modern engineering applications. A key learning outcome was the ability to translate physical system behavior into mathematical models derived from first principles of physics, and to use these models as a basis for controller design and analysis.

Prior to starting the project realization, the evaluation criteria were specified in order to have objective and quantitative parameters to assess whether the project achieved its goals. 
We were able to develop nonlinear and linearized models, which accurately describe the dynamics of the actuated inverted pendulum. Based on these models, PID and LQR controllers were designed, which stabilize the system around the desired equilibrium point. That was a sound basis for further quantitative comparison of both controllers. The comparison was conducted under the identical initial conditions what ensured unbiased results. Consequently, the performance difference was analyzed and presented in the results section of this report. As a result, the evaluation criteria were met.  

As concrete deliverables of this project, we developed MATLAB scripts for a nonlinear and linearized inverted pendulum system actuated by an electromagnet, single-loop and cascaded PID, LQR controller and controller performance comparison script. For cascaded PID and LQR, we created Simulink models of the designed control approaches. Based on these scripts and models, we implemented a 2D interactive animation in MATLAB and 3D animation using a web app in order to visualize the behavior. The accompanying report consolidates the theoretical background, modeling approach, controller design, and quantitative comparison, providing a complete overview of the project accomplishment process.

\end{document}