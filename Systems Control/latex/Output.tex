\documentclass[main.tex]{subfiles}
\begin{document}
\subsection{Output Definition}
Following the linearization, the inverted pendulum system is described by linear state-space model
\begin{equation}
    \dot x = Ax + Bu, x = 
    \begin{bmatrix} 
    \phi \\
    \dot \phi\\
    i
    \end{bmatrix}
\end{equation}
The output equation should be based on available measurements. In this project, the pendulum angle is measured at the pivot point. Therefore, the system output is specified as 
\begin{equation}
    y = \phi
\end{equation}
Since the output corresponds directly to the first state variable, the linear output equation can be written as
\begin{equation}
    y= Cx, C = 
    \begin{matrix}
        [1&0&0]
    \end{matrix}
\end{equation}
The direct feedthrough matrix D is set to zero, since the measured output does not depend directly on the control input. The complete linear state-space representation of the system is therefore given by
\begin{equation}
\left\{ \begin{aligned} 
 \dot x&= Ax + Bx\\
  y &= Cx
  \end{aligned} \right.
\end{equation}
The defined output is used as the feedback signal for the PID controller and for performance evaluation in the LQR-based control scheme.
\end{document}