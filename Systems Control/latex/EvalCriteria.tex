\documentclass[main.tex]{subfiles}
\begin{document}
\section{Evaluation Criteria}
The project's success is evaluated based on objective and quantitative criteria that assess both the stability and performance of the controlled system. Firstly, closed-loop stability is required for all implemented controllers, ensuring that the inverted pendulum can be stabilized around its upright equilibrium point.

Secondly, the performance of the transient response is evaluated using metrics in the time-domain such as the settle time and the overshoot. These parameters indicate how quickly and smoothly the system stabilizes around the desired equilibrium.

Thirdly, the required control effort is assessed using quantitative measures that include the maximum current, the root-mean-square (RMS) current, and the integrated squared current. These metrics provide insight into actuator energy efficiency.

Finally, controller performance is compared under the same model and initial conditions to ensure a fair and consistent evaluation. The project is considered successful if the implemented controllers stabilize the system and allow a quantitative comparison based on specified criteria. 

\subfile{metrics}

\end{document}


